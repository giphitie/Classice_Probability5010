% Options for packages loaded elsewhere
\PassOptionsToPackage{unicode}{hyperref}
\PassOptionsToPackage{hyphens}{url}
%
\documentclass[
]{article}
\usepackage{amsmath,amssymb}
\usepackage{iftex}
\ifPDFTeX
  \usepackage[T1]{fontenc}
  \usepackage[utf8]{inputenc}
  \usepackage{textcomp} % provide euro and other symbols
\else % if luatex or xetex
  \usepackage{unicode-math} % this also loads fontspec
  \defaultfontfeatures{Scale=MatchLowercase}
  \defaultfontfeatures[\rmfamily]{Ligatures=TeX,Scale=1}
\fi
\usepackage{lmodern}
\ifPDFTeX\else
  % xetex/luatex font selection
\fi
% Use upquote if available, for straight quotes in verbatim environments
\IfFileExists{upquote.sty}{\usepackage{upquote}}{}
\IfFileExists{microtype.sty}{% use microtype if available
  \usepackage[]{microtype}
  \UseMicrotypeSet[protrusion]{basicmath} % disable protrusion for tt fonts
}{}
\makeatletter
\@ifundefined{KOMAClassName}{% if non-KOMA class
  \IfFileExists{parskip.sty}{%
    \usepackage{parskip}
  }{% else
    \setlength{\parindent}{0pt}
    \setlength{\parskip}{6pt plus 2pt minus 1pt}}
}{% if KOMA class
  \KOMAoptions{parskip=half}}
\makeatother
\usepackage{xcolor}
\usepackage[margin=1in]{geometry}
\usepackage{color}
\usepackage{fancyvrb}
\newcommand{\VerbBar}{|}
\newcommand{\VERB}{\Verb[commandchars=\\\{\}]}
\DefineVerbatimEnvironment{Highlighting}{Verbatim}{commandchars=\\\{\}}
% Add ',fontsize=\small' for more characters per line
\usepackage{framed}
\definecolor{shadecolor}{RGB}{248,248,248}
\newenvironment{Shaded}{\begin{snugshade}}{\end{snugshade}}
\newcommand{\AlertTok}[1]{\textcolor[rgb]{0.94,0.16,0.16}{#1}}
\newcommand{\AnnotationTok}[1]{\textcolor[rgb]{0.56,0.35,0.01}{\textbf{\textit{#1}}}}
\newcommand{\AttributeTok}[1]{\textcolor[rgb]{0.13,0.29,0.53}{#1}}
\newcommand{\BaseNTok}[1]{\textcolor[rgb]{0.00,0.00,0.81}{#1}}
\newcommand{\BuiltInTok}[1]{#1}
\newcommand{\CharTok}[1]{\textcolor[rgb]{0.31,0.60,0.02}{#1}}
\newcommand{\CommentTok}[1]{\textcolor[rgb]{0.56,0.35,0.01}{\textit{#1}}}
\newcommand{\CommentVarTok}[1]{\textcolor[rgb]{0.56,0.35,0.01}{\textbf{\textit{#1}}}}
\newcommand{\ConstantTok}[1]{\textcolor[rgb]{0.56,0.35,0.01}{#1}}
\newcommand{\ControlFlowTok}[1]{\textcolor[rgb]{0.13,0.29,0.53}{\textbf{#1}}}
\newcommand{\DataTypeTok}[1]{\textcolor[rgb]{0.13,0.29,0.53}{#1}}
\newcommand{\DecValTok}[1]{\textcolor[rgb]{0.00,0.00,0.81}{#1}}
\newcommand{\DocumentationTok}[1]{\textcolor[rgb]{0.56,0.35,0.01}{\textbf{\textit{#1}}}}
\newcommand{\ErrorTok}[1]{\textcolor[rgb]{0.64,0.00,0.00}{\textbf{#1}}}
\newcommand{\ExtensionTok}[1]{#1}
\newcommand{\FloatTok}[1]{\textcolor[rgb]{0.00,0.00,0.81}{#1}}
\newcommand{\FunctionTok}[1]{\textcolor[rgb]{0.13,0.29,0.53}{\textbf{#1}}}
\newcommand{\ImportTok}[1]{#1}
\newcommand{\InformationTok}[1]{\textcolor[rgb]{0.56,0.35,0.01}{\textbf{\textit{#1}}}}
\newcommand{\KeywordTok}[1]{\textcolor[rgb]{0.13,0.29,0.53}{\textbf{#1}}}
\newcommand{\NormalTok}[1]{#1}
\newcommand{\OperatorTok}[1]{\textcolor[rgb]{0.81,0.36,0.00}{\textbf{#1}}}
\newcommand{\OtherTok}[1]{\textcolor[rgb]{0.56,0.35,0.01}{#1}}
\newcommand{\PreprocessorTok}[1]{\textcolor[rgb]{0.56,0.35,0.01}{\textit{#1}}}
\newcommand{\RegionMarkerTok}[1]{#1}
\newcommand{\SpecialCharTok}[1]{\textcolor[rgb]{0.81,0.36,0.00}{\textbf{#1}}}
\newcommand{\SpecialStringTok}[1]{\textcolor[rgb]{0.31,0.60,0.02}{#1}}
\newcommand{\StringTok}[1]{\textcolor[rgb]{0.31,0.60,0.02}{#1}}
\newcommand{\VariableTok}[1]{\textcolor[rgb]{0.00,0.00,0.00}{#1}}
\newcommand{\VerbatimStringTok}[1]{\textcolor[rgb]{0.31,0.60,0.02}{#1}}
\newcommand{\WarningTok}[1]{\textcolor[rgb]{0.56,0.35,0.01}{\textbf{\textit{#1}}}}
\usepackage{graphicx}
\makeatletter
\def\maxwidth{\ifdim\Gin@nat@width>\linewidth\linewidth\else\Gin@nat@width\fi}
\def\maxheight{\ifdim\Gin@nat@height>\textheight\textheight\else\Gin@nat@height\fi}
\makeatother
% Scale images if necessary, so that they will not overflow the page
% margins by default, and it is still possible to overwrite the defaults
% using explicit options in \includegraphics[width, height, ...]{}
\setkeys{Gin}{width=\maxwidth,height=\maxheight,keepaspectratio}
% Set default figure placement to htbp
\makeatletter
\def\fps@figure{htbp}
\makeatother
\setlength{\emergencystretch}{3em} % prevent overfull lines
\providecommand{\tightlist}{%
  \setlength{\itemsep}{0pt}\setlength{\parskip}{0pt}}
\setcounter{secnumdepth}{-\maxdimen} % remove section numbering
\ifLuaTeX
  \usepackage{selnolig}  % disable illegal ligatures
\fi
\usepackage{bookmark}
\IfFileExists{xurl.sty}{\usepackage{xurl}}{} % add URL line breaks if available
\urlstyle{same}
\hypersetup{
  pdftitle={Homework3},
  pdfauthor={Gifty Osei},
  hidelinks,
  pdfcreator={LaTeX via pandoc}}

\title{Homework3}
\author{Gifty Osei}
\date{2024-09-30}

\begin{document}
\maketitle

\subsection{Question 3c}\label{question-3c}

\begin{Shaded}
\begin{Highlighting}[]
\CommentTok{\# Parameters}
\NormalTok{n }\OtherTok{\textless{}{-}} \DecValTok{1000}

\CommentTok{\# Case 1: Small p}
\NormalTok{p1 }\OtherTok{\textless{}{-}} \FloatTok{0.001}
\NormalTok{lambda1 }\OtherTok{\textless{}{-}}\NormalTok{ n }\SpecialCharTok{*}\NormalTok{ p1  }\CommentTok{\# lambda = 1}
\NormalTok{k1 }\OtherTok{\textless{}{-}} \FunctionTok{floor}\NormalTok{(lambda1 }\SpecialCharTok{+} \FloatTok{0.5} \SpecialCharTok{*} \FunctionTok{sqrt}\NormalTok{(lambda1))  }\CommentTok{\# k = 1}

\CommentTok{\# Case 2: Larger p}
\NormalTok{p2 }\OtherTok{\textless{}{-}} \FloatTok{0.01}
\NormalTok{lambda2 }\OtherTok{\textless{}{-}}\NormalTok{ n }\SpecialCharTok{*}\NormalTok{ p2  }\CommentTok{\# lambda = 10}
\NormalTok{k2 }\OtherTok{\textless{}{-}} \FunctionTok{floor}\NormalTok{(lambda2 }\SpecialCharTok{+} \FloatTok{0.5} \SpecialCharTok{*} \FunctionTok{sqrt}\NormalTok{(lambda2))  }\CommentTok{\# k = 11}



\CommentTok{\# Binomial CDFs}
\NormalTok{cdf\_binom\_case1 }\OtherTok{\textless{}{-}} \FunctionTok{pbinom}\NormalTok{(k1, }\AttributeTok{size =}\NormalTok{ n, }\AttributeTok{prob =}\NormalTok{ p1)}
\NormalTok{cdf\_binom\_case2 }\OtherTok{\textless{}{-}} \FunctionTok{pbinom}\NormalTok{(k2, }\AttributeTok{size =}\NormalTok{ n, }\AttributeTok{prob =}\NormalTok{ p2)}

\CommentTok{\# Poisson CDFs}
\NormalTok{cdf\_poisson\_case1 }\OtherTok{\textless{}{-}} \FunctionTok{ppois}\NormalTok{(k1, }\AttributeTok{lambda =}\NormalTok{ lambda1)}
\NormalTok{cdf\_poisson\_case2 }\OtherTok{\textless{}{-}} \FunctionTok{ppois}\NormalTok{(k2, }\AttributeTok{lambda =}\NormalTok{ lambda2)}
\end{Highlighting}
\end{Shaded}

\begin{center} For Case 1:Small $p (p = 0.001, \lambda = 1)$, Binomial CDF at $ k=1$ is 0.7357589 and Poisson CDF at $k=1$ is 0.7357589.

For Case 2: Larger $p (p = 0.01, \lambda = 10)$, Binomial CDF at $k=2$ is 0.6973501 and Poisson CDF at $k=2$ is 0.6967761.

\end{center}

\subsection{Plot to show Difference}\label{plot-to-show-difference}

\begin{Shaded}
\begin{Highlighting}[]
\FunctionTok{library}\NormalTok{(ggplot2)}

\CommentTok{\# Set parameters for Binomial and Poisson distributions}
\NormalTok{n }\OtherTok{\textless{}{-}} \DecValTok{1000}
\NormalTok{lambda\_values }\OtherTok{\textless{}{-}} \FunctionTok{c}\NormalTok{(}\DecValTok{5}\NormalTok{, }\DecValTok{50}\NormalTok{)  }\CommentTok{\# Small and large values of lambda}
\NormalTok{k\_values }\OtherTok{\textless{}{-}} \FunctionTok{c}\NormalTok{(}\FunctionTok{floor}\NormalTok{(}\DecValTok{5} \SpecialCharTok{+} \FloatTok{0.5} \SpecialCharTok{*} \FunctionTok{sqrt}\NormalTok{(}\DecValTok{5}\NormalTok{)), }\FunctionTok{floor}\NormalTok{(}\DecValTok{50} \SpecialCharTok{+} \FloatTok{0.5} \SpecialCharTok{*} \FunctionTok{sqrt}\NormalTok{(}\DecValTok{50}\NormalTok{)))  }\CommentTok{\# Values for k}

\CommentTok{\# Create a data frame to store CDF values}
\NormalTok{results }\OtherTok{\textless{}{-}} \FunctionTok{data.frame}\NormalTok{(}\AttributeTok{k =} \FunctionTok{integer}\NormalTok{(), }\AttributeTok{CDF =} \FunctionTok{numeric}\NormalTok{(), }\AttributeTok{Distribution =} \FunctionTok{character}\NormalTok{(), }\AttributeTok{Lambda =} \FunctionTok{numeric}\NormalTok{())}

\CommentTok{\# Calculate CDF for Binomial and Poisson distributions}
\ControlFlowTok{for}\NormalTok{ (lambda }\ControlFlowTok{in}\NormalTok{ lambda\_values) \{}
\NormalTok{  p }\OtherTok{\textless{}{-}}\NormalTok{ lambda }\SpecialCharTok{/}\NormalTok{ n  }\CommentTok{\# Probability of success for Binomial}
\NormalTok{  k\_range }\OtherTok{\textless{}{-}} \DecValTok{0}\SpecialCharTok{:}\DecValTok{50}  \CommentTok{\# Range of k values to calculate CDF}

  \CommentTok{\# Calculate Binomial CDF}
\NormalTok{  binom\_cdf }\OtherTok{\textless{}{-}} \FunctionTok{pbinom}\NormalTok{(k\_range, n, p)}
\NormalTok{  results }\OtherTok{\textless{}{-}} \FunctionTok{rbind}\NormalTok{(results, }\FunctionTok{data.frame}\NormalTok{(}\AttributeTok{k =}\NormalTok{ k\_range, }\AttributeTok{CDF =}\NormalTok{ binom\_cdf, }\AttributeTok{Distribution =} \StringTok{"Binomial"}\NormalTok{, }\AttributeTok{Lambda =}\NormalTok{ lambda))}
  
  \CommentTok{\# Calculate Poisson CDF}
\NormalTok{  poisson\_cdf }\OtherTok{\textless{}{-}} \FunctionTok{ppois}\NormalTok{(k\_range, lambda)}
\NormalTok{  results }\OtherTok{\textless{}{-}} \FunctionTok{rbind}\NormalTok{(results, }\FunctionTok{data.frame}\NormalTok{(}\AttributeTok{k =}\NormalTok{ k\_range, }\AttributeTok{CDF =}\NormalTok{ poisson\_cdf, }\AttributeTok{Distribution =} \StringTok{"Poisson"}\NormalTok{, }\AttributeTok{Lambda =}\NormalTok{ lambda))}
\NormalTok{\}}

\CommentTok{\# Plotting the CDF comparison}
\FunctionTok{ggplot}\NormalTok{(results, }\FunctionTok{aes}\NormalTok{(}\AttributeTok{x =}\NormalTok{ k, }\AttributeTok{y =}\NormalTok{ CDF, }\AttributeTok{color =}\NormalTok{ Distribution)) }\SpecialCharTok{+}
  \FunctionTok{geom\_line}\NormalTok{() }\SpecialCharTok{+}
  \FunctionTok{geom\_point}\NormalTok{() }\SpecialCharTok{+}
  \FunctionTok{facet\_wrap}\NormalTok{(}\SpecialCharTok{\textasciitilde{}}\NormalTok{ Lambda, }\AttributeTok{scales =} \StringTok{"free\_y"}\NormalTok{, }\AttributeTok{ncol =} \DecValTok{2}\NormalTok{, }\AttributeTok{labeller =} \FunctionTok{labeller}\NormalTok{(}\AttributeTok{Lambda =} \ControlFlowTok{function}\NormalTok{(x) }\FunctionTok{paste}\NormalTok{(}\StringTok{"λ ="}\NormalTok{, x))) }\SpecialCharTok{+}
  \FunctionTok{labs}\NormalTok{(}\AttributeTok{title =} \StringTok{"CDF Comparison of Binomial and Poisson Approximations"}\NormalTok{,}
       \AttributeTok{x =} \StringTok{"k"}\NormalTok{, }\AttributeTok{y =} \StringTok{"CDF"}\NormalTok{) }\SpecialCharTok{+}
  \FunctionTok{theme\_bw}\NormalTok{() }\SpecialCharTok{+}
  \FunctionTok{theme}\NormalTok{(}\AttributeTok{legend.position =} \StringTok{"top"}\NormalTok{)}
\end{Highlighting}
\end{Shaded}

\includegraphics{HW3_files/figure-latex/plot-1.pdf}

The Poisson approximation to the Binomial distribution is more accurate
when \(\lambda\) (or \(p\)) is small, as illustrated in the left plot.
When \(\lambda\) is large, the Poisson approximation is still useful,
but its accuracy diminishes as \(p\) increases or \(\lambda\) increases.
Therefore, The distribution \(Y_n\) converges to the distribution of \$Y
\sim \$ Poisson(\(\lambda\)).

\end{document}
